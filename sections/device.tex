\section{基于设备树的灵活设备发现}

\begin{frame}
    \frametitle{设备管理现状观察}

    \begin{enumerate}
        \item 竞赛操作系统
              \begin{itemize}
                  \item ArceOS:实现了简单的设备发现和管理
                  \item 大部分其他 OS:硬编码
              \end{itemize}
        \item 教学操作系统
              \begin{itemize}
                  \item xv6: 硬编码
                  \item rCore: 实现了简单的设备发现和管理
              \end{itemize}
        \item 需要管理的设备
              \begin{itemize}
                  \item CPU
                  \item SoC (PLIC, PLL, Power)
                  \item UART
                  \item SD 卡控制器
                  \item 网卡
              \end{itemize}

    \end{enumerate}

\end{frame}

\begin{frame}[fragile]
    \frametitle{设备管理实现}

    \begin{enumerate}
        \item 设备管理:Device Trait
              \begin{itemize}
                  \item probe
                  \item interrupt\_handler
              \end{itemize}
        \item 设备树解析
              \begin{itemize}
                  \item 同一份二进制可以在不同的硬件上启动
              \end{itemize}
        \item 设备管理器
    \end{enumerate}

    \begin{lstlisting}
        pub struct DeviceManager {
            cpus: Vec<cpu::CPU>,
            plic: plic::PLIC,
            devices: Vec<Arc<dyn Device>>,
            interrupt_map: BTreeMap<usize, Arc<dyn Device>>,
        }
    \end{lstlisting}


\end{frame}

