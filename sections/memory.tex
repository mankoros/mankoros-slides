\section{任意地址内核装载}


\begin{frame}
    \frametitle{内核装载现状观察}

    \begin{enumerate}
        \item 简易操作系统
              \begin{itemize}
                  \item 固定链接高位地址,汇编打开翻译
                  \item 硬编码 boot 页表到内核数据段
              \end{itemize}
        \item 教学操作系统 xv6-riscv
              \begin{itemize}
                  \item 内核使用低地址空间
                  \item 不能支持大用户程序
              \end{itemize}
        \item Linux
              \begin{itemize}
                  \item 支持装载到 \strong{任意 2M 对齐的地址} 并启动
                  \item 支持内核地址空间随机化 (KASLR)
              \end{itemize}
    \end{enumerate}

\end{frame}

\begin{frame}
    \frametitle{实现难点}

    \begin{enumerate}
        \item 不能硬编码 Boot 页表:需要 Runtime 初始化
        \item 打开翻译前内核位于低地址:需要严格位置无关代码
        \item Boot 页表的内存管理困难:内核的内存管理不可用
        \item 外设尚未初始化:没有外设,没有串口,调试困难
    \end{enumerate}

\end{frame}

\begin{frame}
    \frametitle{实现方法}

    \begin{enumerate}
        \item 汇编初始化物理 stack
        \item rust 初始化页表
              \begin{itemize}
                  \item 填充粗粒度页表 (2 MiB),使用 data 段
                  \item 映射内核映像
                  \item 映射 DTB
              \end{itemize}
        \item 返回汇编
        \item 汇编打开分页,重新初始化虚拟 stack
        \item 进入 rust main boot 流程 (SBI debug console 可用)
              \begin{itemize}
                  \item 解析 DTB
                  \item 映射设备
              \end{itemize}

    \end{enumerate}


\end{frame}
